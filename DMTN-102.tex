

\documentclass[DM,authoryear,toc]{lsstdoc}
% lsstdoc documentation: https://lsst-texmf.lsst.io/lsstdoc.html

% Package imports go here.
\usepackage{graphicx}
\usepackage{url}
\usepackage{latexsym}
\usepackage{color}
\usepackage{enumitem}

% Local commands go here.

% To add a short-form title:
% \title[Short title]{Title}
\title[Alerts Key Numbers]{LSST Alerts: Key Numbers}

% Optional subtitle
% \setDocSubtitle{A subtitle}

\author{%
M.~L.~Graham, E.~Bellm, C.~T.~Slater, L.~Guy, et al.
}

\setDocRef{DMTN-102}

\date{\today}

% Optional: name of the document's curator
% \setDocCurator{The Curator of this Document}
\setDocUpstreamLocation{\url{https://github.com/lsst-dm/dmtn-102}}

\setDocAbstract{%
Abstract text.
}

% Change history defined here.
% Order: oldest first.
% Fields: VERSION, DATE, DESCRIPTION, OWNER NAME.
% See LPM-51 for version number policy.
\setDocChangeRecord{%
  \addtohist{1}{2018-12-06}{Unreleased.}{Melissa Graham}
}

\begin{document}

% Create the title page.
% Table of contents is added automatically with the "toc" class option.
\maketitle

% % % % % % % % % % % % % % % % % % % % % % % % % % % % % % % % % % % %
\section{Introduction} \label{sec:intro}

{\bf This document is currently in development and contains a lot of detailed information for the purposes of internal discussion; to be reduced/clarified later.} \textcolor{red}{Text in red needs some discussion by DM-SST.}

The LSST Data Management System's (DMS) Alert Production (AP) pipeline will process new data as it is obtained by the telescope. Difference Imaging Analysis (DIA) will be performed, and all sources with a signal-to-noise ratio {\tt transSNR} $>5$ in positive or negative flux will be considered ``detected", instantiate a record in the source catalogs, and generate an alert (LSR-REQ-0101 \citeds{LSE-29}; DMS-REQ-0269, -0274 \citeds{LSE-61}). Each alert is a packet containing LSST data about the source such as coordinates, photometry, and image cutouts. For a full description of detected sources and alert packet contents, see \citeds{LSE-163}. The LSST alert stream will be delivered to several community-developed brokers, and also accessible to users via the LSST Science Platform's alert filtering service (AFS); plans and policies for alert distribution are provided in \citeds{LDM-612}. 

The purpose of this document is to quantitatively inform broker developers, and the broader scientific community planning to use alerts, on the key numbers regarding alert generation, distribution, and access via the LSST alert filtering service. The goals of this document are threefold: (1) to provide all of the key numbers regarding alert generation in one place; (2) to include any and all basis information, assumptions, and derivations that contributed to the key number; and (3) to be clear about whether each key number represents an estimate, a requirement, or a boundary. 

In this work we use 8 bits per byte ($\rm B$), and $1024$ $\rm B$ per $\rm KB$, $1024$ $\rm KB$ per $\rm MB$, and so forth.

A list of resources being used in the preparation of this document.
\begin{itemize}
\item {\it LSST: From Science Drivers to Reference Design and Anticipated Data Products}, \citet{2008arXiv0805.2366I}
\item LSST Science Requirements Document (SRD), \citeds{LPM-17}.
\item LSST System Requirements (LSR), \citeds{LSE-29}.
\item Observatory System Specifications (OSS) document, \citeds{LSE-30}.
\item Data Management System Requirements (DMSR) document, \citeds{LSE-61}.
\item Science Requirements and System Specifications Spreadsheet (SR\&SSS), \citeds{LSE-81}.
\item Data Products Definitions Document (DPDD), \citeds{LSE-163}
\item Plans and Policies for LSST Alert Distribution, \citeds{LDM-612}
\item Data Management Science Pipelines Design, \citeds{LDM-151}
\end{itemize}

% % % % % % % % % % % % % % % % % % % % % % % % % % % % % % % % % % % %
\section{Alert Stream} \label{sec:alerts}

The concept and existence of the LSST alert stream is baked into the highest-level document: the SRD specifies that information about the detections of transient, variable, and moving objects be released as a data stream within $1$ minute. 


% % % % % % % % % % % % %
\subsection{Alert Release Timescale}\label{ssec:OTT1}

It is a formal requirement that the Data Management System release $98\%$ of alerts for each visit within $60$ seconds of the end of image readout.

{\bf Formal Requirements --} Regarding the DMS's ability to generate alerts, the SRD states that {\it ``data on likely optical transients ... will be released with a latency of at most {\tt OTT1} minutes"}, where {\tt OTT1} is $1$/$2$/$0.5$ minutes (design/minimum specifications and stretch goal; \citeds{LPM-17}). The SRD's design specification value of {\tt OTT1} flows down to a formal requirement in the LSR that specifies {\it ``LSST shall meet the following specification for reporting of data on optical transients detected in single-visit data [in] {\tt OTT1}"}, and defines {\tt OTT1} as {\it ``the latency of reporting optical transients following the completion of readout of the last image of a visit"} (LSR-REQ-0101, \citeds{LSE-29}). The LSR also makes a formal requirement that the DMS's {\it ``algorithm shall be applied and the alert transmitted within the specified latency for at least a fraction {\tt OTR1} of instances ... [and] remaining transients ... be identified and recorded at the next processing opportunity"}, where {\tt OTR1} is $98\%$ (LSR-REQ-0025, \citeds{LSE-29}). The LSR requirement also flows down to more specific formal requirement in the OSS and DMSR, which both state that {\it ``Alerts shall be made available within time {\tt OTT1} from the conclusion of readout of the raw exposures used to generate each alert to the distribution of the alert to community distribution mechanisms"} (OSS-REQ-0127, \citeds{LSE-30}; DMS-REQ-0004, \citeds{LSE-61}).

There is a mild amount of {\bf tension in the formal requirements}: the SRD states that {\tt OTT1} is a requirement on the {\it releasing} of an alert; the LSR states that {\tt OTT1} is a requirement on the {\it reporting} of an alert (LSR-REQ-0101), and then later that it is the time for {\it transmitting} the alert to the broker (LSR-REQ-0025). It is only the OSS and DMSR which interpret {\tt OTT1} as ending at the time of {\it distribution to the broker}. In this respect, {\bf the OSS and DMSR are incorrect:} the time it will take for an alert to be transmitted to the broker is independent of the LSST Prompt Processing system (see \S~\ref{ssec:num_brokers}), and {\tt OTT1} should end at the time the alert is {\it available to the stream}. A further tension is that whereas {\tt OTT1} is flowed down from the LSR to the OSS and DMSR, {\tt OTR1} is not (see further discussion in \S~\ref{ssec:OTR1}).

\textcolor{red}{Potential Change to the OSS and DMSR:} \\
OSS-REQ-0127 \citedsp{LSE-30} and DMS-REQ-0004 \citedsp{LSE-61} could have this statement updated from: \\
{\it ``Alerts shall be made available within time {\tt OTT1} from the conclusion of readout of the raw exposures used to generate each alert to the distribution of the alert to community distribution mechanisms."} \\
to instead say: \\
{\it ``Alerts shall be made available within time {\tt OTT1} from the conclusion of readout of the raw exposures used to generate each alert. The acceptable fraction of alerts released within a latency of {\tt OTT1} is {\tt OTR1}."}\\
In the DMSR, {\tt OTR1} could be added to the table in Section 1.3.1 ({\it ``Nightly Data Accessible Within 24 hrs"}), which also contains {\tt OTT1}. In the OSS, {\tt OTT1} is used but not defined, so presumably {\tt OTR1} could be used but not defined, too. Another {\tt OTR1}-related change to the OSS is proposed in \S~\ref{ssec:OTR1}.

{\bf Additional Notes --} We often colloquially say {\it OTT1} starts {\it at the time of shutter close}, but it actually starts $2$ seconds later {\it at the end of readout} (or longer if slower reads are adopted).

\textcolor{red}{Requirement Breach Protocol --} For the {\tt OTR1} $=2\%$ of alerts allowed to be transmitted with $>$ {\tt OTT1} seconds, how will they be identified as delayed? I.e., with a flag in the alert packet?

\textcolor{red}{Add some Science Drivers?} What are the main science driveres behind OTT1 = 60 seconds? If OTT1 is descoped to, e.g., 2 minutes, what is the science impact? The SRD states {\it ``Time scales ranging from $\sim1$ min (to constrain the properties of fast faint transients such as those recently discovered by the Deep Lens Survey)..."} are required for science -- check what these fast faint transients are.


% % % % % % % % % % % % %
\subsection{Number of Alerts per Visit}\label{ssec:transN}

It is a formal requirement that the Data Management System can sustain the generation of an average of at least $10000$ alerts per visit. 

%%% Informed by Gregory's comments on the Confluence page
%%% https://confluence.lsstcorp.org/pages/viewpage.action?pageId=94765182
{\bf Formal Requirements --} Regarding the DMS's ability to generate alerts, the SRD states that {\it ``The system should be capable of reporting such data for at least {\tt transN} candidate transients per field of view and visit,"} where {\tt transN} is $10^4$/$10^3$/$10^5$ (design/minimum specifications and stretch goal; \citeds{LPM-17}). The design specification value of {\tt transN} flows down to a formal requirement in the LSR, which describes {\tt transN} as {\it ``the minimum number of optical transients for which data can be reported per visit"}, with a note that {\it ``it is unclear whether the SRD specification of {\tt transN} refers to the number of alerts that can be generated for a single visit (i.e. an instantaneous limit), or the number per visit averaged over time"} (LSR-REQ-0101, \citeds{LSE-29}). The OSS does clarify: {\it ``The LSST Data Management system shall be sized to accommodate an average value of at least {\tt nAlertVisitAvg} alerts generated per standard visit while meeting all its other requirements,"} where {\tt nAlertVisitAvg} is $10^4$ (OSS-REQ-0193 in \citeds{LSE-30}). No corresponding requirement is flowed down to the DMSR, but the SR\&SSS also uses the same minimum average value, and furthermore uses $40000$ as the {\it peak} number of alerts per visit, with a note that it was chosen as a number in between the SRD's minimum of $10^4$ and the stretch goal of $10^5$ \citedsp{LSE-81}. 

There is a mild amount of {\bf tension in the formal requirements}, in that neither {\tt transN} nor {\tt nAlertVisitAvg} are flowed down to the DMSR. 

\textcolor{red}{Potential Change to the DMSR:} \\
If {\tt nAlertVisitAvg} (which is essentially equivalent to {\tt transN}, but more specific) should exist in the DMSR, then perhaps a new sub-section should be added to Section 2.2.3 {\it ``Transient Alert Distribution"} such as: \\
Section 2.2.3.1 ``Performance Requirements for Transient Alert Distribution" \\
ID: DMS-REQ-XXXX \\ 
Specification: The system shall be able to identify and release an average of at least {\tt nAlertVisitAvg} alerts per visit. \\
Table: ``Average minimum number of alerts per visit required to be supported.", $10000$, alerts, {\tt nAlertVisitAvg}. \\
Derived from Requirements: LSR-REQ-0101, OSS-REQ-0193

{\bf Science Drivers --} The value of {\tt transN} $=10000$ alerts per visit is a formal requirement on the DMS, not a scientific estimate of the intrinsic rate of transients and variables in the universe. The number of alerts is expected to be lower/higher in extra/galactic fields. As derived from the Science Book \citep{2009arXiv0912.0201L}, the estimates for the most common transients and variables are as follows:
% extragal: 9.6 * 135e6 * 0.2 * 0.1 / (18000 * 0.8) = 1800 alerts
% galactic: 9.6 *135e6 * 0.8 * 0.1 / (18000 * 0.2)   = 28800 alerts
\begin{itemize}
\item Variable Stars: LSST is predicted to observe a total of $\sim135$ million variable stars. Making the simple assumption that $20$/$80$\% are in extra/galactic fields, and that of the $\sim18000$ $\rm deg^2$ surveyed by LSST $80$/$20$\% of the fields are extra/galactic, and that $10$\% of all variable stars are detectably variable at any given time, then a typical extra/galactic field would yield $\sim1800$/$28800$ alerts per visit. Averaged over all fields, and weighted by $80$/$20$\% of the fields being extra/galactic, this is $7200$ alerts per visit.
\item Supernovae: LSST is predicted to observe a total of 10 million supernovae in 10 years, or 1 million per year. Since SNe are typically only visible for a few months, there might be $\sim0.3$ million detectable at any given time. Over $15000\ deg^{2}$ of extragalactic survey area, that's $\sim20\ {\rm SNe\ deg^{-2}}$ or $\sim200\ {\rm SNe}$ per visit.
\item Active Galactic Nuclei: LSST is predicted to observe millions of AGN. If $\sim10\%$ of them are detectably variable at any given time, then the estimate is $\sim0.1$ million alerts over $15000\ {\rm deg^2}$ would generate $\sim7\ {\rm alerts\ deg^{-2}}$ or $\sim70\ {\rm alerts}$ per visit for AGN.
\item Moving Objects: The predicted number of Solar System objects that LSST is predicted to observe is dominated by the 5.5 million main-belt asteroids. Assuming that they are spread evenly over the $\sim18000$ $\rm deg^2$ survey area (they're not, as they're found primarily along the ecliptic) leads to $\sim3000$ alerts per visit due to moving objects.
\end{itemize} 
Therefore, astrophysical estimates for the occurrence rates of alerts caused by the most common types of transients and variables yield $\sim5100$/$32000$ alerts per visit in extra/galactic fields, with an average of $\sim10500$ alerts per visit.

\textcolor{red}{Requirement Breach Protocol --} Is there a new upper limit on the peak volume? In cases where there are $>40000$ alerts generated by a visit, does DM expect they issued with a delay?

% % % % % % % % % % % % %
\subsection{Alert Packet Size}\label{ssec:packet_size}

There are no formal requirements regarding the alert packet size. The size of an individual alert packet is estimated to be $\lesssim82$ $\rm KB$ (without schema or compression). \textcolor{red}{Eric is working on improved alert packet simulations for a more accurate sizing estimate.}

% (30x30) x  2(32+32+16) = 144000 bits = 18000 bytes = 17.6 kB
{\bf Sizing Estimate --} Alert packet contents will include all of the LSST science data for the triggering detection, including a $\sim12$ month historical record of detections, plus image stamp cutouts. Alert contents are described in more detail in Section 3.5 of \citeds{LSE-163}. Simulated alert packets based on the Apache Avro format are at most $\sim82$/$126$ $\rm KB$, without/with the schema, respectively. This volume represents an alert packet for a variable star with a full $12$ month history of detections. \textcolor{red}{For example, a new unassociated source with only a single detection would be $\sim?$ $\rm KB$, and a $\sim1$ month long transient followed by $\sim11$ months of forced photometry would be $\sim?$ $\rm KB$.} The application of gzip compression can further reduce the size of a $\sim82$/$126$ $\rm KB$ alert to $\sim68$/$65$ $\rm KB$ (JIRA ticket DM-16280). Cutout stamps included in the alert will be at least $30\times30$ pixels and contain flux (32 bit/pix), variance (32 bit/pix), and mask (16 bit/pix) extensions for both the template and difference image, plus a header of metadata \citedsp{LSE-163}. The stamps alone will contribute $\gtrsim18$ $\rm KB$ to the total size of the uncompressed alert packet (i.e., $\sim20\%$).

{\bf Alternative ``Lite" Content Options --} Brokers which plan to do their own source association, compile source catalogs based on alerts, or not use the image stamps might prefer a stream of packets with appropriately reduced information. The LSST DM team currently expects that some options will be possible, and brokers may propose an option that works for them during the selection process \citedsp{LDM-612}. As previously mentioned, removing the image stamps would reduce packet size by $\gtrsim18$ $\rm KB$. Removing the historical records of past detections would reduce all alert packets to be equivalent in size to a new unassociated source. A few of these options might also be available to users of the LSST alert filtering service (\S~\ref{sec:LAFS}). 


% % % % % % % % % % % % %
\subsection{Alert Stream Data Rate}\label{ssec:data_rate}

There are no formal requirements regarding the alert stream data rate. The {\it time-averaged} data rate of the alert stream is estimated to be $\sim27$ $\rm MB/sec$ (for alert packets without schema or compression), potentially with bursts of up to $640$ $\rm MB/sec$.

% 82 KB/alert * 10000 alerts/visit = 820000 KB/visit = 800 MB/visit
% 800 MB/visit released in:
%  60 seconds is 13.3 MB/s
%  30 seconds is 26.7 MB/s
%  5 seconds is 160 MB/s

{\bf Sizing Estimate --} The size of a single LSST alert will be $\sim82$ $\rm KB$ (including image stamps but not schema nor compression). Using an average of $\sim10000$ alerts released per $\sim30$ second image, this leads to a {\it time-averaged} alert stream data rate of  $\sim27$ $\rm MB\ s^{-1}$. As discussed in \S~\ref{ssec:transN}, the number of alerts per field will vary in extra/galactic fields from $\sim2000$ to $\lesssim40000$, which would produce {\it time-averaged} alert streams of $\sim5.4$ to $\lesssim108$ $\rm MB\ s^{-1}$. However, in order to release alerts within {\tt OTT1} $=60$ seconds of image readout (\S~\ref{ssec:OTT1}), the stream will not be continuous in time, but periodic, with potential bursts. For example, if all $10000$ alerts are issued within the last $5$ seconds of {\tt OTT1} this would produce a data rate of $160$ $\rm MB\ s^{-1}$; in galactic fields with $\lesssim40000$ alerts this could be as high as $640$ $\rm MB\ s^{-1}$.

\textcolor{red}{Quoting a Data Rate for Filtered Streams --} If pre-filtering will be offered, it might be useful to quote stream rates for the expected common filtering. For example, if a broker wants to limit its incoming flow to an alert stream with {\tt transCompletenessMin} $=90\%$ and {\tt transPurityMin} $=95\%$ (i.e., apply {\tt spuriousness} threshold {\tt T}; \S~\ref{ssec:comp_pure}), then given the purity of the {\tt transSNR} $>5$ stream is only $\sim50\%$ (\S~\ref{ssec:comp_pure} and \citeds{LSE-81}) this might decrease the stream by $\sim40\%$. 



% % % % % % % % % % % % %
\subsection{Number of Selected Brokers}\label{ssec:num_brokers}

There are no formal requirements on the number of brokers. The DM team estimates that resources will allow for the delivery of the alert stream to $4$-$7$ brokers.

{\bf Formal Requirements --} Neither the SRD, LSR, OSS, nor the DMSR place any formal requirements on the number of brokers that the alert stream should be delivered to, and the DPDD and the SR\&SSS do not mention an estimate for the number of brokers. 

{\bf Sizing Estimate --} As described in Section 2.2.3 of \citeds{LDM-612}, {\it ``An allocation of 10 Gbps is baselined for alert stream transfer from the LDF, with an estimated packet size of 82 KB and up to 10,000 alerts per visit. For illustration, based on these numbers up to 7 brokers could receive the full stream if 5 seconds is budgeted for outbound data transfer."} \textcolor{red}{Is there another document to cite for the allocation of 10 Gbps?}


% % % % % % % % % % % % %
\subsection{Alert Database Volume}\label{ssec:adb_volume}

There are no formal requirements on the alerts database volume. The estimated maximum upper limit is $\lesssim3$ $\rm PB$ (without schema or compression).

% 10,000,000 alerts per night * 82 KB/alert = 820,000,000 KB per night = 782 GB/night
{\bf Sizing Estimate --} An upper estimate is derived by starting with a maximum of $\sim1000$ visits per night, and $\sim10000$ alerts per visit, which amounts to $\sim10$ million alerts per night. Using $2000$/$36000$ alerts per visit for extra/galactic fields and assuming $80$/$20$\% of the $1000$ visits per night are for extra/galactic fields, as described in \S~\ref{ssec:transN}, also generates $\sim 10$ million alerts per night. Assuming the upper estimate of $\sim82$ $\rm KB$ per alert (\S~\ref{ssec:packet_size}), that leads to a total of $\sim782$ $\rm GB$ per night. An extreme upper limit is $365$ nights per year for 10 years, which would amount to $\sim2.7$ $\rm PB$ {\it at the very most}. Therefore we quote an extreme upper limit on the alerts database as $\lesssim3$ $\rm PB$. Compression could drastically lower this, as could reformatting: every alert contains a $\sim12$ month historical record and links to the most recent DIAObject and DR Object catalogs. The set of alerts for the same transient/variable would contain significant redundant information which could be reformatted (i.e., removed and compiled). 

% % % % % % % % % % % % %
\subsection{Delayed/Failed Alert Distribution}\label{ssec:OTR1}

It is a formal requirement that $98\%$ of all alerts for a given visit are issued within $60$ seconds.

It is a formal requirement that $<1\%$ of all science visits have any fraction of their alerts experience a distribution delay $>60$ seconds.

It is a formal requirement that $<0.1\%$ of all science visits experience a total failure in alert generation and distribution.

{\bf Formal Requirements --} The SRD does not say anything on the topic of alert distribution delays or failures. As mentioned in \S~\ref{ssec:OTT1}, the LSR defines {\tt OTR1} as the {\it ``fraction of detectable alerts for which an alert is actually transmitted within latency {\tt OTT1}"}, where {\tt OTR1} $=98\%$ (LSR-REQ-0025; \citeds{LSE-29}). The OSS does not state any requirements on the fraction of failed alerts per visit, but does specify that {\it ``no more than {\tt sciVisitAlertFailure} \% of science visits ... shall fail to be subjected to alert generation and distribution"}, where {\tt sciVisitAlertFailure} $=0.1\%$, and that {\it ``no more than {\tt sciVisitAlertDelay} \% of science visits ... shall have their alert generation and distribution completed later than [{\tt OTT1}]"}, where {\tt sciVisitAlertDelay} $=1\%$ (OSS-REQ-0112; \citeds{LSE-30}). The OSS furthermore makes the distinction that if any number of the alerts for a given visit are distributed later than {\tt OTT1}, it counts towards {\it sciVisitAlertDelay}. The DMSR makes no statements about the fraction of alerts per visit with delayed/failed distribution, or the fraction of visits with failed/delayed alert distribution.

There is some {\bf tension in the formal requirements}, as the OSS does not use {\tt OTR1} ($=98\%$, the acceptable fraction of alerts per visit released within {\tt OTT1}) to define a failed/delayed visit. Instead, if a visit has any number of alerts delayed beyond {\tt OTT1} it is considered as counting towards {\tt sciVisitAlertDelay}. Basically, LSR-REQ-0025 and {\tt OTR1} do not appear to have flowed down to the OSS or DMSR, and {\tt OTR1} does not seem to actually be used for anything (unless it's flowed down directly to science verification documents). Furthermore, there is no equivalent to {\tt sciVisitAlertFailure} and/or {\tt sciVisitAlertDelay} in the DMSR. Finally, there is no specification for an acceptable fraction of alerts which completely fail to be distributed (i.e., a requirement like {\tt OTR1}, but for the fraction of alerts with a failed instead of a delayed release).

\textcolor{red}{Potential Change to the OSS:}\\
The definition of {\tt sciVisitAlertDelay} in the OSS could be updated with the blue text to include {\tt OTR1}, such as: \\ 
Section 3.1.2.1 {\it ``Science Visit Alert Generation Reliability"} \\
ID: OSS-REQ-0112 \\
Specification: No more than {\tt sciVisitAlertDelay} \% of science visits read out in the camera [and specified to be analyzed by Data Management] shall have their alert generation and distribution completed later than the SRD specification for alert latency {\tt OTT1}; \textcolor{blue}{this applies to {\tt OTR1} \% of the alerts from that visit as specified in the LSR.} \\
Note: {\tt OTT1} is used but not defined in the OSS, and {\tt OTR1} is not yet defined in the OSS either.

\textcolor{red}{Potential Change to the DMSR}: \\
If {\tt sciVisitAlertFailure} and/or {\tt sciVisitAlertDelay} should be included as formal requirements in the DMSR, then perhaps a new sub-section could be added to Section 2.2.3 {\it ``Transient Alert Distribution"} such as: \\
Section 2.2.3.1 ``Alert Delay and Failure Tolerances" \\
ID: DMS-REQ-???? \\
Specification: Of the science visits read out of the camera and specified to be analyzed by Data Management, no more than {\tt sciVisitAlertDelay} $=1\%$ and {\tt sciVisitAlertFailure} $=0.1\%$ shall have their alert generation and release delayed or failed, respectively (integrated over all stages of data handling). \\
Derived from Requirements: OSS-REQ-0112

\textcolor{red}{Potential Change to the LSR:}\\
Should there be a specification which is like {\tt OTR1} (which specifies that $=98\%$ of alerts per visit must not be delayed beyond {\tt OTT1}) but specifies the acceptable fraction of alerts per visit which fail to be released? Should this just be $0\%$? Or $0.1\%$ like {\tt sciVisitAlertFailure}?

\textcolor{red}{Requirement Breach Protocol --} Are alerts from a delayed visit flagged in some way? With regards to {\tt OTR1} $=98\%$, LSR-REQ-0025 states that {\it `` the remaining transients so detectable must still be identified and recorded at the next processing opportunity"}, but this is not this flowed down to DMSR, and it is unclear what {\it ``the next processing opportunity"} means.


% % % % % % % % % % % % %
\subsection{Alert Stream Completeness and Purity}\label{ssec:comp_pure}

It is a formal requirement that DM derive and supply threshold values for a {\it spuriousness} parameter (also known as {\it real/bogus}), which can be used to filter all alerts into a subsample with a specified completeness and purity, and reduce the fraction of false positives per visit (i.e., sources detected that are not astrophysical in origin).

{\bf Formal Requirements --} The SRD makes no statements about alert stream purity or completeness, but does quote that the {\it ``minimum signal-to-noise ratio in difference image for reporting detection of a transient object"} has a design specification of {\tt transSNR} $=5$ \citedsp{LPM-17}. The LSR contains essentially the same definition for {\tt transSNR}, {\it ``the signal-to-noise ratio in single-visit difference images above which all optical transients are to be reported"} (LSR-REQ-0101; \citeds{LSE-29}). There is no minimum specification or stretch goal associated with {\tt transSNR}. The OSS requires that {\it ``there shall exist a spuriousness threshold {\tt T} for which the completeness and purity of selected difference sources are higher than {\tt transCompletenessMin} and {\tt transPurityMin}, respectively, at the SNR detection threshold {\tt transSampleSNR}. This requirement is to be interpreted as an average over the entire survey"} (OSS-REQ-0353; \citeds{LSE-30}). In other words, the DMS must be able to provide the value for a spuriousness threshold {\tt T}, below which all difference sources detected with a signal-to-noise ratio {\tt transSampleSNR} $=6$, over the entire LSST survey, have {\tt transCompletenessMin} $=90\%$ and {\tt transPurityMin} $=95\%$. This is not the same as a formal requirement on the fraction of false positives per visit in the alert stream, but the spuriousness threshold {\tt T} will allow users to filter their stream to a fiducial completeness and purity. The DMSR does not appear to have any requirements on the fraction of false positives. 

{\bf The SR\&SSS estimates that the number of alerts per visit due to false positives will be $5050$ alerts, or $\sim50\%$ of all alerts \citedsp{LSE-81}.} Based on a footnote on page 19 of \citeds{LDM-151}, which states that {\it ``50\% false positive rate is given in the OSS (when discussing Solar System Object requirements) and impacts the sizing model for the alert stream"}, it appears that this stems from the OSS specification that {\it ``There shall exist a spuriousness threshold {\tt T} for which the completeness and purity of difference sources are higher than {\tt mopsCompletenessMin} and {\tt mopsPurityMin}, respectively, at the SNR detection threshold {\tt orbitObservationThreshold}. This requirement is intended to be interpreted as an average for any one month of observing"}, where {\tt orbitObservationThreshold} $=5$, {\tt mopsCompletenessMin} $=99\%$, and {\tt mopsPurityMin} $=50\%$ (OSS-REQ-0354; \citeds{LSE-30}).


% % % % % % % % % % % %
% \subsection{Number of New Transients per Visit}
% 
% Estimate unclear. Need for quoting it here also unclear (except, it is called out in the spreadsheet).
% 
% \textcolor{red}{MLG thoughts --} This would refer to new, unassociated {\tt DIASources} which are not false. The SR\&SSS quotes the {\it ``average number of alerts per visit due to new transients"} as 100, with no reference (Science Book maybe?). Here we could quote, for example, the average number of unassociated DIASources in either a new WFD extragalactic image or a new Galactic Plane image of a field that has not been observed in $3$ days. Or the predicted number of unassociated DIASources which are actually new/first detections of moving objects. Brainstorm other useful quantities for brokers.



% % % % % % % % % % % % % % % % % % % % % % % % % % % % % % % % % % % %
\section{The LSST Alert Filtering Service} \label{sec:LAFS}

It is a formal requirement that the LSST provide a simple alerts filtering service for users (individuals with LSST data rights and access to the Science Platform), which is hereafter referred to as the LSST alert filtering service (AFS).

{\bf Formal Requirements --} The SRD specifies that {\it ``users will have an option of a query-like pre-filtering of [the alert] data stream in order to select likely candidates for specific transient type"} and that {\it ``several pre-defined filters optimized for traditionally popular transients, such as supernovae and microlensed sources, will also be available"} \citedsp{LPM-17}. Neither the LSR nor the OSS have a formal requirement on this capability, as it is a product of the DMS. The DMSR has a formal requirement that {\it ``a basic, limited capacity, alert filtering service shall be provided that can be given user defined filters to reduce the alert stream to manageable levels"}, and that this service include {\it ``a predefined set of simple filters"} (DMS-REQ-0342, -0348; \citeds{LSE-61}). 


% % % % % % % % % % % % %
\subsection{Number of Simultaneous AFS Users}\label{ssec:AFS_users}

It is a formal requirement that the AFS support a minimum of $100$ simultaneous users.

{\bf Formal Requirements --} The DMSR specifies that the LSST {\it ``alert filtering service shall support {\tt numBrokerUsers} simultaneous users"}, where {\tt numBrokerUsers} $=100$ (DMS-REQ-0343; \citeds{LSE-61}).

\textcolor{red}{MLG: I can't find the background numbers that drive this limit of 100.}

% % % % % % % % % % % % %
\subsection{Number of Alerts per Visit Returned per User-Defined Filter}\label{ssec:AFS_returns}

It is a formal requirement that the AFS return $20$ alerts per visit per user.

{\bf Formal Requirements --} The DMSR specifies that within the LSST alert filtering service {\it ``each user [shall be] allocated a bandwidth capable of receiving the equivalent of {\tt numBrokerAlerts} alerts per visit"}, where {\tt numBrokerAlerts} $=20$ (DMS-REQ-0343; \citeds{LSE-61}).

\textcolor{red}{Requirement Drivers --} Note that in a footnote of \citeds{LDM-612}, it says that the {\it ``requirement on the number of simultaneously connected users and number of passed alerts is largely driven by outbound bandwidth limitations from the DAC at NCSA. We are investigating approaches that would support larger numbers of active filters"} (page 12; \citeds{LDM-612}). MLG: I can't find any references to LAFS capacities in the SR\&SSS.

% % % % % % % % % % % % %
\subsection{Alerts Database Query Latency}

It is a formal requirement that {\it ``All published transient alerts ... shall be available for query"} (OSS-REQ-0185; \citeds{LSE-30}). Like all other Prompt data products, the Alerts Database will be updated within {\tt L1PublicT} $=24$ hours.

{\bf Formal Requirements --} \textcolor{red}{Just point to the L1PublicT requirement.}


% % % % % % % % % % % % % % % % % % % % % % % % % % % % % % % % % % % %
\clearpage

% Include all the relevant bib files.
% https://lsst-texmf.lsst.io/lsstdoc.html#bibliographies
\bibliography{local,lsst,lsst-dm,refs_ads,refs,books}

\end{document}
