\documentclass[DM,authoryear,toc]{lsstdoc}
% lsstdoc documentation: https://lsst-texmf.lsst.io/lsstdoc.html

% Package imports go here.
\usepackage{graphicx}
\usepackage{url}
\usepackage{latexsym}
\usepackage{color}
\usepackage{enumitem}

% Local commands go here.
% DO NOT EDIT - generated by /Users/melissagraham/latex/lsst-texmf/bin/generateAcronyms.py from https://lsst-texmf.lsst.io/.
\newacronym{AP} {AP} {Alert Production}
\newacronym{AURA} {AURA} {\gls{Association of Universities for Research in Astronomy}}
\newglossaryentry{Alert} {name={Alert}, description={A packet of information for each source detected with signal-to-noise ratio > 5 in a difference image during Prompt Processing, containing measurement and characterization parameters based on the past 12 months of LSST observations plus small cutouts of the single-visit, template, and difference images, distributed via the internet}}
\newglossaryentry{Alert Production} {name={Alert Production}, description={The principal component of Prompt Processing that processes and calibrates incoming images, performs Difference Image Analysis to identify DIASources and DIAObjects, packages and distributes the resulting Alerts, and runs Solar System Processing.}}
\newglossaryentry{Alternative Standard Visit} {name={Alternative Standard Visit}, description={A single observation of an LSST field comprised of one 30 second exposure}}
\newglossaryentry{Archive} {name={Archive}, description={The repository for documents required by the NSF to be kept. These include documents related to design and development, construction, integration, test, and operations of the LSST observatory system. The archive is maintained using the enterprise content management system DocuShare, which is accessible through a link on the project website www.project.lsst.org}}
\newglossaryentry{Archive Center} {name={Archive Center}, description={Part of the LSST Data Management System, the LSST archive center is a data center at NCSA that hosts the LSST Archive, which includes released science data and metadata, observatory and engineering data, and supporting software such as the LSST Software Stack}}
\newglossaryentry{Association Pipeline} {name={Association Pipeline}, description={An application that matches detected Sources or DIASources or generated Objects to an existing catalog of Objects, producing a (possibly many-to-many) set of associations and a list of unassociated inputs. Association Pipelines are used in Prompt Processing after DIASource generation and in the final stages of Data Release processing to ensure continuity of Object identifiers}}
\newglossaryentry{Association of Universities for Research in Astronomy} {name={Association of Universities for Research in Astronomy}, description={ consortium of US institutions and international affiliates that operates world-class astronomical observatories, AURA is the legal entity responsible for managing what it calls independent operating Centers, including LSST, under respective cooperative agreements with the National Science Foundation. AURA assumes fiducial responsibility for the funds provided through those cooperative agreements. AURA also is the legal owner of the AURA Observatory properties in Chile}}
\newacronym{B} {B} {Byte (8 bit)}
\newglossaryentry{Center} {name={Center}, description={An entity managed by AURA that is responsible for execution of a federally funded project}}
\newglossaryentry{Construction} {name={Construction}, description={The period during which LSST observatory facilities, components, hardware, and software are built, tested, integrated, and commissioned. Construction follows design and development and precedes operations. The LSST construction phase is funded through the \gls{NSF} \gls{MREFC} account}}
\newacronym{DCR} {DCR} {Differential Chromatic Refraction}
\newacronym{DIA} {DIA} {Difference Image Analysis}
\newglossaryentry{DIAObject} {name={DIAObject}, description={A DIAObject is the association of DIASources, by coordinate, that have been detected with signal-to-noise ratio greater than 5 in at least one difference image. It is distinguished from a regular Object in that its brightness varies in time, and from a SSObject in that it is stationary (non-moving)}}
\newglossaryentry{DIASource} {name={DIASource}, description={A DIASource is a detection with signal-to-noise ratio greater than 5 in a difference image}}
\newacronym{DM} {DM} {\gls{Data Management}}
\newacronym{DMS} {DMS} {Data Management Subsystem}
\newacronym{DMTN} {DMTN} {DM Technical Note}
\newacronym{DR} {DR} {Data Release}
\newacronym{DRP} {DRP} {Data Release Production}
\newglossaryentry{Data Management} {name={Data Management}, description={The LSST Subsystem responsible for the Data Management System (DMS), which will capture, store, catalog, and serve the LSST dataset to the scientific community and public. The DM team is responsible for the DMS architecture, applications, middleware, infrastructure, algorithms, and Observatory Network Design. DM is a distributed team working at LSST and partner institutions, with the DM Subsystem Manager located at LSST headquarters in Tucson}}
\newglossaryentry{Data Management Subsystem} {name={Data Management Subsystem}, description={The Data Management Subsystem is one of the four subsystems which constitute the LSST Construction Project. The Data Management Subsystem is responsible for developing and delivering the LSST Data Management System to the LSST Operations Project}}
\newglossaryentry{Data Management System} {name={Data Management System}, description={The computing infrastructure, middleware, and applications that process, store, and enable information extraction from the LSST dataset; the DMS will process peta-scale data volume, convert raw images into a faithful representation of the universe, and archive the results in a useful form. The infrastructure layer consists of the computing, storage, networking hardware, and system software. The middleware layer handles distributed processing, data access, user interface, and system operations services. The applications layer includes the data pipelines and the science data archives' products and services}}
\newglossaryentry{Data Release} {name={Data Release}, description={The approximately annual reprocessing of all LSST data, and the installation of the resulting data products in the LSST Data Access Centers, which marks the start of the two-year proprietary period}}
\newglossaryentry{Data Release Production} {name={Data Release Production}, description={An episode of (re)processing all of the accumulated LSST images, during which all output DR data products are generated. These episodes are planned to occur annually during the LSST survey, and the processing will be executed at the Archive Center. This includes Difference Imaging Analysis, generating deep Coadd Images, Source detection and association, creating Object and Solar System Object catalogs, and related metadata}}
\newglossaryentry{Difference Image} {name={Difference Image}, description={Refers to the result formed from the pixel-by-pixel difference of two images of the sky, after warping to the same pixel grid, scaling to the same photometric response, matching to the same PSF shape, and applying a correction for Differential Chromatic Refraction. The pixels in a difference thus formed should be zero (apart from noise) except for sources that are new, or have changed in brightness or position. In the LSST context, the difference is generally taken between a visit image and template. }}
\newglossaryentry{Difference Image Analysis} {name={Difference Image Analysis}, description={The detection and characterization of sources in the Difference Image that are above a configurable threshold, done as part of Alert Generation Pipeline}}
\newglossaryentry{Differential Chromatic Refraction} {name={Differential Chromatic Refraction}, description={The refraction of incident light by Earth's atmosphere causes the apparent position of objects to be shifted, and the size of this shift depends on both the wavelength of the source and its airmass at the time of observation. DCR corrections are done as a part of DIA}}
\newglossaryentry{DocuShare} {name={DocuShare}, description={The trade name for the enterprise management software used by LSST to archive and manage documents}}
\newglossaryentry{Document} {name={Document}, description={Any object (in any application supported by DocuShare or design archives such as PDMWorks or GIT) that supports project management or records milestones and deliverables of the LSST Project}}
\newacronym{FITS} {FITS} {\gls{Flexible Image Transport System}}
\newglossaryentry{Flexible Image Transport System} {name={Flexible Image Transport System}, description={an international standard in astronomy for storing images, tables, and metadata in disk files. See the IAU FITS Standard for details}}
\newacronym{GB} {GB} {Gigabyte}
\newglossaryentry{Handle} {name={Handle}, description={The unique identifier assigned to a document uploaded to DocuShare}}
\newacronym{IAU} {IAU} {International Astronomical Union}
\newglossaryentry{JIRA} {name={JIRA}, description={issue tracking product (not an acronym but a truncation of Gojira the Japanese name for Godzilla)}}
\newacronym{KB} {KB} {KiloByte}
\newacronym{LDM} {LDM} {LSST Data Management (Document Handle)}
\newacronym{LPM} {LPM} {LSST Project Management (Document Handle)}
\newacronym{LSE} {LSE} {LSST Systems Engineering (Document Handle)}
\newacronym{LSST} {LSST} {Large Synoptic Survey Telescope}
\newacronym{MB} {MB} {MegaByte}
\newacronym{MREFC} {MREFC} {\gls{Major Research Equipment and Facility Construction}}
\newglossaryentry{Major Research Equipment and Facility Construction} {name={Major Research Equipment and Facility Construction}, description={the NSF account through which large facilities construction projects such as LSST are funded}}
\newacronym{NCSA} {NCSA} {National Center for Supercomputing Applications}
\newacronym{NSF} {NSF} {\gls{National Science Foundation}}
\newglossaryentry{National Science Foundation} {name={National Science Foundation}, description={primary federal agency supporting research in all fields of fundamental science and engineering; NSF selects and funds projects through competitive, merit-based review}}
\newglossaryentry{Non-Standard Visit} {name={Non-Standard Visit}, description={Any single observation of a LSST field that is not comprised of either two 15 second 'Snap' exposures (a standard visit) or one 30 second exposure (an alternative standard visit). For example, exposure times for Special Programs might be significantly shorter or longer than a standard visit (or of random length)}}
\newacronym{OSS} {OSS} {Observatory System Specifications; LSE-30}
\newglossaryentry{Object} {name={Object}, description={In LSST nomenclature this refers to an astronomical object, such as a star, galaxy, or other physical entity. E.g., comets, asteroids are also Objects but typically called a Moving Object or a Solar System Object (SSObject). One of the DRP data products is a table of Objects detected by LSST which can be static, or change brightness or position with time}}
\newglossaryentry{Operations} {name={Operations}, description={The 10-year period following construction and commissioning during which the LSST Observatory conducts its survey}}
\newacronym{PB} {PB} {PetaByte}
\newacronym{PSF} {PSF} {Point Spread Function}
\newglossaryentry{Project Manager} {name={Project Manager}, description={The person responsible for exercising leadership and oversight over the entire LSST project; he or she controls schedule, budget, and all contingency funds}}
\newglossaryentry{Prompt Processing} {name={Prompt Processing}, description={The processing that occurs at the Archive Center on the nightly stream of raw images coming from the telescope, including Difference Imaging Analysis, Alert Production, and Solar System Processing. This processing generates Prompt Data Products.}}
\newglossaryentry{Release} {name={Release}, description={Publication of a new version of a document, software, or data product. Depending on context, releases may require approval from Project- or DM-level change control boards, and then form part of the formal project baseline}}
\newacronym{SSP} {SSP} {Solar System Processing}
\newglossaryentry{Science Pipelines} {name={Science Pipelines}, description={The library of software components and the algorithms and processing pipelines assembled from them that are being developed by DM to generate science-ready data products from LSST images. The Pipelines may be executed at scale as part of LSST Prompt or Data Release processing, or pieces of them may be used in a standalone mode or executed through the LSST Science Platform. The Science Pipelines are one component of the LSST Software Stack}}
\newglossaryentry{Science Platform} {name={Science Platform}, description={A set of integrated web applications and services deployed at the LSST Data Access Centers (DACs) through which the scientific community will access, visualize, and perform next-to-the-data analysis of the LSST data products}}
\newglossaryentry{Snap} {name={Snap}, description={One 15 second exposure within a Standard Visit in the LSST cadence}}
\newglossaryentry{Software Stack} {name={Software Stack}, description={Often referred to as the LSST Stack, or just The Stack, it is the collection of software written by the LSST Data Management Team to process, generate, and serve LSST images, transient alerts, and catalogs. The Stack includes the LSST Science Pipelines, as well as packages upon which the DM software depends. It is open source and publicly available}}
\newglossaryentry{Solar System Object} {name={Solar System Object}, description={A solar system object is an astrophysical object that is identified as part of the Solar System: planets and their satellites, asteroids, comets, etc. This class of object had historically been referred to within the LSST Project as Moving Objects}}
\newglossaryentry{Solar System Processing} {name={Solar System Processing}, description={Solar System Processing (SSP) identifies new SSObjects using unassociated DIASources. SSP is part of the Science Pipelines.}}
\newglossaryentry{Source} {name={Source}, description={A single detection of an astrophysical object in an image, the characteristics for which are stored in the Source Catalog of the DRP database. The association of Sources that are non-moving lead to Objects; the association of moving Sources leads to Solar System Objects. (Note that in non-LSST usage "source" is often used for what LSST calls an Object.)}}
\newglossaryentry{Standard Visit} {name={Standard Visit}, description={A single observation of a LSST field comprised of two 15 second 'Snap' exposures that are immediately combined. An 'Alternative Standard Visit' is a single observation of a LSST field comprised of one 30 second exposure}}
\newglossaryentry{Subsystem} {name={Subsystem}, description={A set of elements comprising a system within the larger LSST system that is responsible for a key technical deliverable of the project}}
\newglossaryentry{Subsystem Manager} {name={Subsystem Manager}, description={responsible manager for an LSST subsystem; he or she exercises authority, within prescribed limits and under scrutiny of the Project Manager, over the relevant subsystem's cost, schedule, and work plans}}
\newglossaryentry{Systems Engineering} {name={Systems Engineering}, description={an interdisciplinary field of engineering that focuses on how to design and manage complex engineering systems over their life cycles. Issues such as requirements engineering, reliability, logistics, coordination of different teams, testing and evaluation, maintainability and many other disciplines necessary for successful system development, design, implementation, and ultimate decommission become more difficult when dealing with large or complex projects. Systems engineering deals with work-processes, optimization methods, and risk management tools in such projects. It overlaps technical and human-centered disciplines such as industrial engineering, control engineering, software engineering, organizational studies, and project management. Systems engineering ensures that all likely aspects of a project or system are considered, and integrated into a whole}}
\newacronym{US} {US} {United States}
\newglossaryentry{Visit} {name={Visit}, description={A sequence of one or more consecutive exposures at a given position, orientation, and filter within the LSST cadence. See \gls{Standard Visit}, \gls{Alternative Standard Visit}, and \gls{Non-Standard Visit}}}
\newglossaryentry{airmass} {name={airmass}, description={The pathlength of light from an astrophysical source through the Earth's atmosphere. It is given approximately by sec z, where z is the angular distance from the zenith (the point directly overhead, where airmass = 1.0) to the source}}
\newglossaryentry{astronomical object} {name={astronomical object}, description={A star, galaxy, asteroid, or other physical object of astronomical interest. Beware: in non-LSST usage, these are often known as sources}}
\newacronym{deg} {deg} {degree; unit of angle}
\newglossaryentry{flux} {name={flux}, description={Shorthand for radiative flux, it is a measure of the transport of radiant energy per unit area per unit time. In astronomy this is usually expressed in cgs units: erg/cm2/s}}
\newglossaryentry{metadata} {name={metadata}, description={General term for data about data, e.g., attributes of astronomical objects (e.g. images, sources, astroObjects, etc.) that are characteristics of the objects themselves, and facilitate the organization, preservation, and query of data sets. (E.g., a FITS header contains metadata)}}
\newglossaryentry{pipeline} {name={pipeline}, description={A configured sequence of software tasks (Stages) to process data and generate data products. Example: Association Pipeline}}
\newglossaryentry{shape} {name={shape}, description={In reference to a Source or Object, the shape is a functional characterization of its spatial intensity distribution, and the integral of the shape is the flux. Shape characterizations are a data product in the DIASource, DIAObject, Source, and Object catalogs}}
\newglossaryentry{transient} {name={transient}, description={A transient source is one that has been detected on a difference image, but has not been associated with either an astronomical object or a solar system body}}

\makeglossaries

% To add a short-form title:
% \title[Short title]{Title}
\title[Alerts Key Numbers]{LSST Alerts: Key Numbers}

% Optional subtitle
% \setDocSubtitle{A subtitle}

\author{%
M.~L.~Graham, E.~Bellm, L.~Guy, C.~T.~Slater, G.~Dubois-Felsmann, and the \gls{Data Management System} Science Team
}

\setDocRef{DMTN-102}

\date{\today}

% Optional: name of the document's curator
% \setDocCurator{The Curator of this Document}
\setDocUpstreamLocation{\url{https://github.com/lsst-dm/dmtn-102}}

\setDocAbstract{%
A quantitative review of the key numbers associated with the \gls{LSST} \gls{Alert} Stream.
}

% Change history defined here.
% Order: oldest first.
% Fields: VERSION, DATE, DESCRIPTION, OWNER NAME.
% See LPM-51 for version number policy.
\setDocChangeRecord{%
  \addtohist{1}{2019-02-19}{Released.}{Melissa Graham}
  \addtohist{1.1}{2019-01-06}{Minor updates, added glossary.}{Melissa Graham}
}

\begin{document}

% Create the title page.
% Table of contents is added automatically with the "toc" class option.
\maketitle

% % % % % % % % % % % % % % % % % % % % % % % % % % % % % % % % %
\section{Introduction} \label{sec:intro}

The \gls{LSST} \gls{Data Management System}'s (\gls{DMS}) \gls{Alert Production} (\gls{AP}) \gls{pipeline} will process new data as it is obtained by the telescope. 
Difference Imaging Analysis (\gls{DIA}) will be performed, and all sources with a signal-to-noise ratio\reqparam{transSNR}\lsrreq{0101}\dmreq{0269}\dmreq{0274} $>$5 (in positive or negative \gls{flux}) will be considered "detected", a record will be instantiated in the source catalogs, and an alert generated.
Each alert is a packet containing \gls{LSST} data about the source, such as coordinates, photometry, and image cutouts. 
For a full description of detected sources and alert packet contents, see \citeds{LSE-163}. 
The \gls{LSST} alert stream will be delivered to several community-developed brokers, and also accessible to users\footnote{In this case, "users" is restricted to individuals with \gls{LSST} data rights and access privileges.} via an alert filtering service through the \gls{LSST} Data Access Centers (DACs). 
Plans and policies for alert distribution are provided in \citeds{LDM-612}. 

The purpose of this document is to quantitatively inform broker developers, and the broader scientific community planning to use alerts, on the key numbers regarding alert generation, distribution, and access via the \gls{LSST} alert filtering service. 
The goals of this document are threefold: (1) to provide all of the key numbers regarding alert generation in one place; (2) to include any and all basis information, assumptions, and derivations that contributed to the key number; and (3) to be clear about whether each key number represents an estimate, a requirement, or a boundary. 
Wherever possible, the reference to a specific \gls{LSST} requirement and any relevant requirement parameters are provided in the right-hand column. 
In this document we use 8 bits per byte (\gls{B}), and 1024~B per \gls{KB}, 1024~KB per \gls{MB}, and so forth.

% The resources used in the preparation of this document are as follows:
% \begin{itemize}
% \item {\it LSST: From Science Drivers to Reference Design and Anticipated Data Products}, \citet{2008arXiv0805.2366I}
% \item LSST Science Requirements Document (SRD), \citeds{LPM-17}.
% \item LSST System Requirements (LSR), \citeds{LSE-29}.
% \item Observatory System Specifications (OSS) document, \citeds{LSE-30}.
% \item Data Management System Requirements (DMSR) document, \citeds{LSE-61}.
% \item Science Requirements and System Specifications Spreadsheet (SR\&SSS), \citeds{LSE-81}.
% \item Data Products Definitions Document (DPDD), \citeds{LSE-163}
% \item Plans and Policies for LSST Alert Distribution, \citeds{LDM-612}
% \item Data Management Science Pipelines Design, \citeds{LDM-151}
% \end{itemize}


% % % % % % % % % % % % % % % % % % % % % % % % % % % % % % % % %
\section{Alert Stream} \label{sec:alerts}

The concept and existence of the \gls{LSST} alert stream was first introduced by the highest-level document, the \gls{LSST} Science Requirements \gls{Document} \citedsp{LPM-17}, which specifies that information about the detections of \gls{transient}, variable, and moving objects be released promptly as a data stream. 

	
% % % % % % % % % % % % %
\subsection{Alert \gls{Release} Timescale}\label{ssec:OTT1}

{\bf It is a requirement that the \gls{DMS} be capable of supporting the distribution of at least 98\%\reqparam{OTT1}\reqparam{OTR1}\lsrreq{0101}\lsrreq{0025}\ossreq{0127}\dmreq{0004} of alerts for each visit within 60 seconds of the end of image readout\footnote{The design, minimum, and stretch values for the alert release timescale are 1, 2, and 0.5 minutes \citedsp{LPM-17}.}.}

This requirement applies to visits resulting in fewer than 40,000 alerts, and the term "distribution" includes all steps up to and including the transmission of the alert packet out of the \gls{LSST} Data Facility (i.e., it does not include the time it takes for a broker to receive or ingest the alert). It is furthermore specified that all delayed alerts be made available at the next opportunity (\citeds{LPM-17}; see also the discussion regarding delayed/failed alert distribution in \S~\ref{ssec:OTR1}).


% % % % % % % % % % % % %
\subsection{Number of Alerts per \gls{Visit} (and per Night)}\label{ssec:transN}

{\bf It is a requirement that the \gls{DMS} support the distribution\reqparam{transN}\lsrreq{0101}\reqparam{nAlertVisitAvg}\ossreq{0193}\reqparam{nAlertVisitPeak}\dmreq{0393} of at least 10,000 alerts per standard visit\footnote{The design, minimum, and stretch values for the number of alerts per visit are $10^4$, $10^3$, and $10^5$ \citedsp{LPM-17}.} on average during a given night, and at least 40,000 alerts per single standard visit.}

An extension of the above is that the \gls{DMS} will support a long-term average number of $10^7$ alerts distributed per night (assuming an average of 1,000 visits per night). It is furthermore specified that the performance of alert distribution shall degrade gracefully beyond these values, meaning that visits resulting in an excess of alerts should not cause any \gls{DMS} downtime.

The value of 10,000 alerts per visit is a requirement on the \gls{DMS} and not a scientific estimate of the intrinsic rate of transients and variables in the universe. However, estimates for the most common transients and variables can be derived from the Science Book \citep{2009arXiv0912.0201L} by making some significant assumptions, as follows:
\begin{itemize}
\item Variable Stars: \gls{LSST} is predicted to observe a total of $\sim$135 million variable stars. Making the simple assumption that 20\% (80\%) of the stars are in extragalactic (Galactic) fields, and that of the $\sim$18,000 \gls{deg}$^2$ surveyed by \gls{LSST}, 80\% (20\%) of the fields are extragalactic (Galactic), and that 10\% of all variable stars are detectably variable at any given time, then a typical extragalactic (Galactic) field would yield $\sim$1,800 ($\sim$28,800) alerts per visit \footnote{Since ``detectably variable" means ``significantly different from the template", the value of 10\% does depend on how the template is generated.}. Averaged over all fields, and weighted by 80\% and 20\% of the fields being extragalactic and Galactic, respectively, this is an average of 7,200 alerts per visit.
\item Supernovae: \gls{LSST} is predicted to observe a total of $\sim$10 million supernovae in 10 years, or $\sim$1 million per year. Since SNe are typically only visible for a few months, there might be $\sim$0.3 million detectable at any given time. Over 15,000 \gls{deg}$^{2}$ of extragalactic survey area, that's $\sim$20 SNe \gls{deg}$^{-2}$ or $\sim$200 alerts for SNe per visit.
\item Active Galactic Nuclei: \gls{LSST} is predicted to observe millions of AGN. If $\sim$10\% of them are detectably variable at any given time, then the estimate is that $\sim$0.1 million alerts over 15,000 \gls{deg}$^2$ would generate $\sim$7 alerts \gls{deg}$^{-2}$, or $\sim$70 alerts per visit for AGN.
\item Moving Objects: The number of Solar System objects that LSST is predicted to observe is dominated by the 5.5 million main-belt asteroids. Due to their concentration along the ecliptic, estimates for the number of moving objects range from $\sim$400 alerts per visit on average but up to $\sim$5000 alerts per visit in the densest areas of the ecliptic.
\end{itemize} 
Therefore, astrophysical estimates for the occurrence rates of alerts caused by the most common types of transients and variables yield $\sim$5,100 ($\sim$32,000) alerts per visit in extragalactic (Galactic) fields, with an average of $\sim$10,500 alerts per visit.


% % % % % % % % % % % % %
\subsection{Alert Packet Size}\label{ssec:packet_size}

{\bf The size of an individual alert packet is estimated to be $\lesssim$82~KB.}

There are no requirements regarding the alert packet size. The statement above is an estimate based on the planned content of the alerts as described in Section 3.5 of \citeds{LSE-163}. Simulated alert packets based on the Apache Avro format are at most $\sim$82~KB. This volume represents an alert packet for a variable star with a full 12 month history of detections, and the history alone accounts for $\sim$27KB of the alert packet ($\sim$33\%). Cutout stamps included in the alert will be at least 30$\times$30 pixels and contain \gls{flux} (32 bit/pix), variance (32 bit/pix), and mask (16 bit/pix) extensions for both the template and difference image, plus a header of \gls{metadata} \citedsp{LSE-163}. The stamps alone will contribute $\gtrsim$18~KB to the total size of the uncompressed alert packet (i.e., $\sim$20\%). The application of gzip compression can further reduce the size of an alert to $\sim$65~KB (\gls{JIRA} ticket \gls{DM}-16280). 

{\bf ``Lite" Packet Options --} Brokers that plan to do their own source association, compile source catalogs based on alerts, or not use the image stamps might prefer a stream of packets with appropriately reduced information. 
The \gls{LSST} \gls{DM} team is currently open to exploring options for supporting ``Lite" versions of alert packets. Individual broker teams may indicate which information they require (or would like removed from the packets in their stream) during the broker proposal process \citedsp{LDM-723}.
As previously mentioned, removing the image stamps would reduce packet size by $\gtrsim$18~KB, and removing the historical records of past detections could reduce packet size by up to $\sim$27~KB. 
A few of these options might also be available to users of the \gls{LSST} alert filtering service (\S~\ref{sec:LAFS}).

% % % % % % % % % % % % %
\subsection{Alert Stream Data Rate}\label{ssec:data_rate}

{\bf The time-averaged data rate of the alert stream is estimated to be $\sim$0.2~Gbps, potentially with bursts of up to 5.4~Gbps.}

There are no requirements regarding the alert stream data rate. The values quoted in the statement above are estimates based on the expected size of an alert packet ($\sim$82~KB, Section \ref{ssec:packet_size}), the number of alerts per visit, and the alert distribution mechanism. Using an average of 10,000 alerts released per standard visit, this leads to a {\it time-averaged} alert stream data rate of  $\sim$0.2~Gbps. As discussed in \S~\ref{ssec:transN}, the number of alerts per field will vary in extragalactic and Galactic fields from $\sim$2,000 to $\lesssim$40,000, which would produce {\it time-averaged} alert streams of $\sim$0.04 to $\lesssim$0.8~Gbps. However, in order to release alerts within 60 seconds of image readout (\S~\ref{ssec:OTT1}), the stream will not be continuous in time, but periodic, with potential bursts: if all 10,000 alerts are issued within the last 5 seconds of that window the data rate would be 1.3~Gbps. In galactic fields with $\sim$40,000 alerts per visit this could be as high as 5.4~Gbps.


% % % % % % % % % % % % %
\subsection{Number of Selected Brokers}\label{ssec:num_brokers}

{\bf It is a requirement\reqparam{numStreams}\dmreq{0391} that the \gls{DMS} be capable of supporting the transmission of at least 5 full alert streams within 60 seconds of image readout.}

This requirement is based in part on what estimates of the alert stream data rate and the bandwidth allocated to alert distribution have shown will be possible to support. 

% % % % % % % % % % % % %
\subsection{Alert Database Volume}\label{ssec:adb_volume}

{\bf The estimated size of the alerts database after 10 years is $\sim$2.2~PB.}

There are no requirements on the alerts database volume. The statement above is an estimate based on the alert packet contents, the number of alerts per night, and the expected number of observing nights per year. As described in \S~\ref{ssec:transN}, the \gls{DMS} system will support an average of $\sim$10 million alerts per night (which approximately matches the expected scientific yields). Assuming the upper estimate of $\sim$82~KB per alert (\S~\ref{ssec:packet_size}), that leads to a total of $\sim$782~GB per night. After accounting for downtime and weather\ossreq{0080}\ossreq{0081}\ossreq{0082} the total number of observing nights is 300 per year \citedsp{LSE-30}, which leads to an estimate of $\sim$2.2~PB after 10 years.


% % % % % % % % % % % % %
\subsection{Delayed/Failed \gls{Alert} Distribution}\label{ssec:OTR1}

{\bf It is a requirement\reqparam{sciVisitAlertDelay}\reqparam{sciVisitAlertFailure}\ossreq{0112}\dmreq{0392} that no more than 1\% of all standard visits fail to have at least 98\% of its alerts distributed within 60 seconds of image readout, and that no more than 0.1\% of all standard visits fail to distribute alerts.}

These requirements apply to standard visits which should have produced $\leq$40,000 alerts. For example, a visit would be considered "delayed" and count towards that 1\% limit if $>$2\% of its alerts were distributed with a latency of $>$60 seconds. The requirement that no more than 0.1\% of all science visits fail to generate and/or distribute alerts is integrated over all stages of data handling, not just alert distribution, and includes failures at any stage of prompt processing.

For an average of 10,000 alerts per visit and 1,000 visits per night, this requirement allows the \gls{DMS} to distribute up to 2\% (200,000 alerts per night) with a latency $>$60 seconds after image readout.
The worst-case scenario for a night of alert distribution which still meets these requirements is if 989 visits all have just under 2\% of their alerts delayed by $>$60 seconds and distributed within 24 hours (197,800 alerts delayed), {\it and} 10 visits (1\%) have all of their alerts distributed with $>$60 seconds and $<$24 hours (100,000 alerts delayed), {\it and} 1 visit (0.1\%) completely fails to generate and/or distribute any alerts (10,000 alerts ).


% % % % % % % % % % % % %
\subsection{Alert Stream Completeness and Purity}\label{ssec:comp_pure}

{\bf It is a requirement that \gls{DM} derive and supply threshold values\reqparam{transSampleSNR}\reqparam{transCompletenessMin}\reqparam{transPurityMin}\ossreq{0353} for a spuriousness\footnote{In this context, spuriousness is like a real/bogus score.} parameter, which can be used to filter alerts into a subsample of \gls{transient} and variable objects with a given completeness and purity.}

The requirement is that \gls{DM} calculate a spuriousness parameter for all alerts, and derive and supply a spuriousness threshold value that filters the full stream into a subsample of alerts for \gls{transient} and variable objects\footnote{See also \gls{OSS}-REQ-0354 for the required parameters for a subsample of \gls{transient} and variable objects \citedsp{LSE-30}.} that is 90\% complete and 95\% pure for all sources with a signal-to-noise ratio $>$6. While the requirements on purity and completeness are specified as point thresholds, \gls{DM} expects to provide information to enable users to choose spuriousness threshold values that can be used to filter the stream to a desired level of completeness and purity, thereby reducing the fraction of false positives (sources detected that are not astrophysical in origin) to a level that is appropriate for their science goals. Brokers could request a pre-filtered stream that includes a restriction on spuriousness.


% % % % % % % % % % % % % % % % % % % % % % % % % % % % % % % % %
\section{The \gls{LSST} Alert Filtering Service} \label{sec:LAFS}

{\bf It is a requirement that the \gls{LSST} provide an alerts filtering service for users.}\dmreq{0342}\dmreq{0348} 

The \gls{LSST} alerts filtering service is a mechanism by which users --- individuals with \gls{LSST} data rights and access --- can receive alerts via pre-defined filters that have been optimized for established \gls{transient} classifications such as supernovae and/or create and apply their own filters to the stream \citedsp{LPM-17,LSE-61}. 


% % % % % % % % % % % % %
\subsection{Number of Simultaneous Users}\label{ssec:LAFS_users}

{\bf It is a requirement that the \gls{LSST} alert filtering service be able to support\reqparam{numBrokerUsers}\dmreq{0343} at least 100 simultaneous users.}

This requirement is driven by outbound bandwidth limitations from the Data Access \gls{Center} at the National \gls{Center} for Supercomputing Applications (\gls{NCSA}); the \gls{DM} team is currently investigating approaches that would support larger numbers of active users \citedsp{LDM-612}. During \gls{LSST} \gls{Operations}, if the total number of simultaneous users is oversubscribed then a proposal process may be instituted \citedsp{LSE-163}.


% % % % % % % % % % % % %
\subsection{Number of Alerts per \gls{Visit} Returned}\label{ssec:LAFS_returns}

{\bf It is a requirement that the \gls{LSST} alert filtering service be able to return 20\reqparam{numBrokerAlerts}\dmreq{0343} full-sized alerts per visit per user.}

Assuming 1,000 visits per night (\S~\ref{ssec:transN}), each user's filter will be capable of returning 20,000 alerts per night, which would amount to $\sim$1.6~GB (\S~\ref{ssec:packet_size}).


% % % % % % % % % % % % %
\section{Alerts \gls{Archive}}\label{ssec:LAFS_adb}

{\bf It is a requirement that all alerts be stored in an archival database and be available for retrieval.}\dmreq{0094}

The term "available for retrieval" applies to users with data rights and access to the \gls{LSST} \gls{Science Platform}. Like all other Prompt data products, the alerts archive will be updated within 24\reqparam{L1PublicT}\lsrreq{0104} hours \citedsp{LSE-29}. The alerts archive is not a part of the \gls{LSST} alert filtering service, but is included in this section to raise awareness of its existence. 

The \gls{LSST} \gls{DM} team anticipates that the alerts archive will support queries by their unique alert identification numbers, but might not support searches by coordinate, time, magnitude, or other alert attributes. For this reason, the alerts archive should {\em not} be considered a viable alternative for users who {\em do} wish to study \gls{transient}, variable and moving objects with the \gls{LSST}, but who {\em do not} require immediate (i.e., same-night) access to sources detected via difference image analysis. In other words, queries to the alerts archival database should not be construed as a viable alternative to community brokers or the \gls{LSST} alert filtering service. Instead, users with science goals that are achievable with a latency of $\geq$24 hours should plan to use the Prompt data products described in Section 3 of \citeds{LSE-163}. Furthermore, users with science goals that are achievable with latencies of a year or more (i.e., archival time-domain studies) should plan to use the Data \gls{Release} data products described in Section 4 of \citeds{LSE-163}.


\appendix
% Include all the relevant bib files.
% https://lsst-texmf.lsst.io/lsstdoc.html#bibliographies
\label{sec:bib}
\bibliography{local,lsst,lsst-dm,refs_ads,refs,books}

\label{sec:acronyms}
\printglossaries

\end{document}
